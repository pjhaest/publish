\chapter{Basic Usage}
\index{usage}

This chapter contains installation instructions. It
also contains basic information on how to use the command-line
interface.
 
\section{Installation}
\index{installation}
\package{} follows the standard installation procedure for Python packages. 
Enter the source directory of \package{} and issue the following command:
\begin{code}
# sudo python setup.py install
\end{code}

See Appendix \ref{installation} for a more detailed description.

\section{Command-Line Interface}
\index{command-line interface}

The synopsis for the \package{} command-line interface looks as follows:

\begin{code}
# publish <command> [command/global option]... [filename]
\end{code}

The command always starts with \emp{publish} followed by a
\emp{<command>} which may be be either \emp{import}, \emp{validate},
or \emp{export}. Import and export can be used with some global
option, such as \emp{year=2008}, or
\emp{author=name\_of\_choice}. Import and export always require
a filename to import from or export to.

Examples:

\begin{code}
# publish import category=books bookfile.bib

# publish export year=2002 output.pdf
\end{code}

The command \emp{validate} can be used by itself or with a file with
suffix \emp{.pub}. If used by itself, it is the default database-file
\emp{papers.pub} that is validated.

Examples:

\begin{code}
# publish validate

# publish validate inputfile.pub
\end{code}

The command \emp{export} may be used to export a publication record,
for example:

\begin{code}
# publish export papers.pdf
\end{code}

This will be discussed in more detail below in Chapters \ref{import}
(importing), \ref{validate} (validation), and \ref{export}
(exporting).

\section{Global Options}
\index{options}

The following global options are recognized by all three commands:
\begin{itemize}
\item
  \emp{debug=yes} \\

  Enable debugging output. With this option, Python exceptions are not
  caught, resulting in a full Traceback on errors.
\item
  \emp{autofix=yes} \\

  Automatically choose the default option when an error is
  encountered. This is useful (but dangerous) when importing large
  amounts of data into the database.
\end{itemize}

\section{Configuration of the System}
\index{configuration}

In the file \emp{publish/config/general.py}, it is possible to make
some configuration options.  For example, it is possible to change the
name of the default database file, the venues-file, and the file were
invalid papers are stored.  It is also possible to change how picky
the system is when judging the closeness between papers that are
considered as duplicates (\emp{matching\_\-distance\_strong}) and how
far the system will look to suggest a venue name when a venue (journal
name) is not recognized (\emp{matching\_distance\_weak}).
