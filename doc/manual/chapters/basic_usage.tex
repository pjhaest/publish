\chapter{Basic Usage}
\index{usage}

This chapter contains installation instructions. It
also contains basic information on how to use the command-line
interface.

\section{Installation}
\index{installation}
\package{} follows the standard installation procedure for Python packages.
Enter the source directory of \package{} and issue the following command:
\begin{code}
# sudo python setup.py install
\end{code}

See Appendix \ref{installation} for a more detailed description.

\section{Command-Line Interface}
\index{command-line interface}

The synopsis for the \package{} command-line interface looks as follows:

\begin{code}
# publish <command> [command/global option]... [filename]
\end{code}

The command always starts with \emp{publish} followed by a
\emp{<command>} which may be be either \emp{import}, \emp{validate},
or \emp{export}. Import and export can be used with some global
option, such as \emp{year=2008}, or
\emp{author=name\_of\_choice}. Import and export always require
a filename to import from or export to.

Examples:

\begin{code}
# publish import category=books bookfile.bib

# publish export year=2002 output.pdf
\end{code}

The command \emp{validate} can be used by itself or with a file with
suffix \emp{.pub}. If used by itself, it is the default database-file
\emp{papers.pub} that is validated.

Examples:

\begin{code}
# publish validate

# publish validate inputfile.pub
\end{code}

The command \emp{export} may be used to export a publication record,
for example:

\begin{code}
# publish export papers.pdf
\end{code}

This will be discussed in more detail below in Chapters \ref{import}
(importing), \ref{validate} (validation), and \ref{export}
(exporting).

\section{Global Options}
\index{options}

The following global options are recognized by all three commands:
\begin{itemize}
\item
  \emp{debug=yes} \\

  Enable debugging output. With this option, Python exceptions are not
  caught, resulting in a full Traceback on errors.
\item
  \emp{autofix=yes} \\

  Automatically choose the default option when an error is
  encountered. This is useful (but dangerous) when importing large
  amounts of data into the database.
\end{itemize}

\section{Configuration of the System}
\index{configuration}

Much of the behavior of the system can be configured by the user.
The file \emp{publish/config/defaults.py} contains default values of
all the variables that can be configured.
This includes, for example, the
name of the default database file, the venues-file, and the file where
invalid papers are stored.  It is also possible to change how picky
the system is when judging the closeness between papers that are
considered as duplicates (\emp{matching\_\-distance\_strong}) and how
far the system will look to suggest a venue name when a venue (journal
name) is not recognized (\emp{matching\_distance\_weak}).

To change the default settings, the user has to make a file
\emp{publish\_config.py} and make sure the \emp{PYTHONPATH} variable
is set such that any \emp{import publish\_config} imports the desired
file. This means that the user may have different \emp{publish\_config.py}
files for different purposes, and the one to be used by \package{}
is controlled by the \emp{PYTHONPATH} variable. In most cases, one
would like to change the default database name or other variables for
a particular project, and then it is sufficient to make
a local \emp{publish\_config.py} file in the directory where \package{}
is supposed to be run from and where the database is to be stored
(there is no need to set \emp{PYTHONPATH} in such cases
since Python will always try
to import from the current
working directory first). Note that the user's configuration file
\emph{must} have the name \emp{publish\_config.py}.

Suppose we want to change the names of the database and the venues list.
For this purpose
we make a \emp{publish\_config.py} file with the following content

\begin{code}
from publish.config.defaults import *

database_filename = '.publish_papers.pub'
local_venues_filename = '.publish_venues.txt'
\end{code}
Note that we start the filenames with a dot to make them hidden in
the project directory.

Sometimes you may want to add words that need special capitalization.
Looking at the \emp{publish/config/defaults.py} file, we realize that
such words are stored in a dictionary \emp{uppercase}. We want to
add capitalization of the words ``FEniCS'' and ``FEniCS-based''.
This means that if those words, however they are capitalized in a title,
will be guaranteed to be written that way. The above file will
now look as follows

\begin{code}
from publish.config.defaults import *

database_filename = '.publish_papers.pub'
local_venues_filename = '.publish_venues.txt'

uppercase.update({
    "fenics": "{FEniCS}",
    "fenics-based": "{FEniCS}-based",
    })
\end{code}
The syntax for specifying capitalization is taken from BibTeX. Note
that the key in the \emp{uppercase} dictionary must be in lowercase.
Also note that we do \emp{uppercase.update} to add new items to
the dictionary. Doing \emp{uppercase = ...} will just assign two
entries and forget about all the entries that the default settings take
care (i.e., we overwrite the \emp{uppercase} dictionary imported in
the first line). Lists and tuples must be \emph{extended} with new
values, typically by an assignment like \emp{obj = obj + users\_list}.

It is possible to skip the first import line in \emp{publish\_config.py}
files. \package{} will then merge the \emp{publish\_config.py}
contents with the default values in \emp{publish/config/defaults.py}.
The syntax changes a little bit in \emp{publish\_config.py} since
the variables to be configured do not exist from an import. For example,
the \emp{uppercase} variable set above is not imported so we just do
an ordinary assignment:

\begin{code}
database_filename = '.publish_papers.pub'
local_venues_filename = '.publish_venues.txt'

uppercase = {
    "fenics": "{FEniCS}",
    "fenics-based": "{FEniCS}-based",
    }
\end{code}
Lists and tuples are not extended; just assign the desired new values.
